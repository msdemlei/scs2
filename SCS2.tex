\documentclass[11pt,a4paper]{ivoa}
\input tthdefs
\input gitmeta

\usepackage{todonotes}

\title{IVOA SCS: Simple Cone Searches for Astronomical Catalogues}

% see ivoatexDoc for what group names to use here; use \ivoagroup[IG] for
% interest groups.
\ivoagroup{DAL}

\author[https://wiki.ivoa.net/twiki/bin/view/IVOA/MarkusDemleitner]{
  Markus Demleitner}

\editor{Markus Demleitner}

% \previousversion[????URL????]{????Concise Document Label????}
\previousversion{This is the first public release}


\begin{document}
\begin{abstract}
The Simple Cone Search Protocol is an IVOA protocol to facilitate
publishing and querying astronomical catalogues.  This document defines
query parameters, response formats, and metadata standards.  It defines
version 2 of this protocol, which is incompatible with the IVOA's SCS
version 1 protocol.  We also give some guidance on transitioning between
the two major versions.
\end{abstract}


\section*{Conformance-related definitions}

The words ``MUST'', ``SHALL'', ``SHOULD'', ``MAY'', ``RECOMMENDED'', and
``OPTIONAL'' (in upper or lower case) used in this document are to be
interpreted as described in IETF standard RFC2119 \citep{std:RFC2119}.

The \emph{Virtual Observatory (VO)} is a
general term for a collection of federated resources that can be used
to conduct astronomical research, education, and outreach.
The \href{https://www.ivoa.net}{International
Virtual Observatory Alliance (IVOA)} is a global
collaboration of separately funded projects to develop standards and
infrastructure that enable VO applications.


\section{Introduction}

In 2008, Simple Cone Search (SCS) version 1 \citep{2008ivoa.specQ0222P}
was one of the first standards in the Virtual Observatory to gain
significant traction; there was a standardised way to query astronomical
catalogues -- tabular data with celestial positions and a simple primary
key -- across data centres.

In addition to taking up previous standards efforts, it was exploring
various technologies, such as simple data model annotation using UCDs,
bespoke error reproting pattersn, and metadata extensions for the VO
Registry; at least within the VO, this was all breaking new ground.

Not surprisingly, in the 18 years since SCS1, many of these approaches
turned out to be insufficient; in particular the completely outdated
form of SCS1's UCDs make it stick out like a sore thumb in today's
Virtual Observatory.  More generally, SCS should respect the general
rules for protocols defined by the Data Access Layer Working Group, DALI
\citep{2017ivoa.spec.0517D}.

Since SCS1's release, the Table Access Protocol TAP
\citep{2019ivoa.spec.0927D} has been defined.  Thus, there is now a far
superior protocol to query tabular data.  However, a nice for a cone
search-like protocol still exists: something that, on the service side,
can be implemented ad-hoc, and on the client side is dead simple to use
provided you have a VOTable parser.

The remainder of this document first explores the use cases in a bit
more detail, then defines the endpoints required for an SCS2 service,
goes on to specify the response format and concludes with considerations
of how to register, discover, and query SCS2 services.

\subsection{Role within the VO Architecture}

\begin{figure}
\centering

% As of ivoatex 1.2, the architecture diagram is generated by ivoatex in
% SVG; copy ivoatex/archdiag-full.xml to role_diagram.xml and throw out
% all lines not relevant to your standard.
% Notes don't generally need this.  If you don't copy role_diagram.xml,
% you must remove role_diagram.pdf from SOURCES in the Makefile.

\includegraphics[width=0.9\textwidth]{role_diagram.pdf}
\caption{Architecture diagram for this document}
\label{fig:archdiag}
\end{figure}

Fig.~\ref{fig:archdiag} shows the role this document plays within the
IVOA architecture \citep{2021ivoa.spec.1101D}.

SCS2 directly makes use of the following standards:

\begin{bigdescription}
\item[DALI] \citet{2017ivoa.spec.0517D} gives the basic rules all modern
DAL protocols have to follow.  We will freely reference DALI in many
places.  The expectation is that any DALI 1 version should yield
compatible services.

\item[VOTable] \citet{2025ivoa.spec.0116O} defines the central response
format for SCS2 services.  Actually, most other possible response
formats are missing metadata required for proper SCS operations; in that
sense, there is no SCS2 without VOTable.

\item[UCD] \citet{2005ivoa.spec.0819D} gives a lightweight semantic
annotation at allows cleints to (semi-) automatically process SCS2
results.  In contrast to its predecessory, however, SCS2 no longer uses
UCDs to locate pieces of data relevant for protocol operation.

\item[VODataService] \citet{2021ivoa.spec.1102D} is the metadata scheme
used to register SCS2 services.

\end{bigdescription}

SCS2 also relates to the following other VO standards:

\begin{bigdescription}
 \item[ConeSearch 1] \citet{2008ivoa.specQ0222P} is the incompatible
predecessor of SCS2.  We give a plan for how to transition from it in
this document.

\item[TAP] \citet{2019ivoa.spec.0927D} is the IVOA's protocol for
running advanced queries against tables of all sorts.  Its existence
allows us to ignore more advanced use cases by defering them to TAP.

\item[DataLink] \citet{2023ivoa.spec.1215B} offers a mechanism to link
advanced data products, in particular parts of a catalogue row's
provenance chain, to rows returned by SCS2.
\end{bigdescription}

\section{Use Cases}

\subsection{Consumer Use Case}

SCS2 is driven by the main use case: Researchers wants to retrieve a
full astronomical catalogue or parts of it and immediately start working
with it, without having to write importing software, and ideally with
support in quickly adapting to different catalogues.  The basic
specification of what part of a catalogue is by position; being able to
formulate further simple constraints is a nice extra, but SCS2 does not
attempt to compete with TAP.  Still, clients should be able to produce
expressiv user interfaces to SCS2 services on the fly.

A related second use case is: Researchers want to locate a catalogue
with certain properties (e.g., the presence of a column covering a
certain sort of physics, some limiting magnitude, spatial or spectral
coverage).  On finding a matching resource, they can immediately access
data pertaining to their research (and preferably not terribly much
more).

To briefly map these use cases to features covered by this standard:

\begin{itemize}
\item Immediately start working: VOTable response format
\item Positional contraint: The mandatory RA, DEC, SR parameters
\item Other parts of catalogues: The possibility to declare further
parameters and their standard DALI interval syntax
\item Locate a catalogue: Our use of the VO Registry, including
registering tablesets containing UCDs.
\end{itemize}


\subsection{Operator Use Case}

A data publisher wants to make one or more catalogues accessible to the
scientific public with minimal effort while still satisfying the
consumer use case.

This important part in this use case is ``with minimal effort''.
Without this requirement, TAP would be the protocol to choose, and data
providers are urged to see whether they can adapt one of the existing
TAP/ADQL implementations.  However, implementing TAP and ADQL is a major
effort.  SCS2's goal is to be implementable within a week or so even
without a relational database.

\section{Endpoints}

SCS2 is what DALI \citep{2017ivoa.spec.0517D} calls a ``concrete service
specification''.  As envisioned in DALI, we will only reference DALI
sections where they apply unchanged.  The references are for DALI 1.2.

All SCS2 services must implement three endpoints with the names given
here:

\begin{itemize}
\item \verb|<base-url>/scs2| -- synchronous query.  This is the query
endpoint further specified below.  Its full URL is what is registered as
``service URL''.  All parameters on this endpoint are case-sensitive.

\item \verb|<base-url>/capabilities| -- VOSI capabilities as per DALI
sect.~2.5.  This must describe the SCS2 and VOSI capabilities but may
contain arbitrary other capabilities.

\item \verb|<base-url>/tables| -- VOSI tables as per DALI sect.~2.6.
The names of all tables in the tableset returned must work as values for
the TABLE parameter of the SCS2 endpoint.
\end{itemize}

Note that this scheme in particular means that a client can reliably
infer the URL of the VOSI capabilities by computing the sibling of the
service URL named ``capabilities''.

\section{Parameters in SCS2}

SCS2 services are required to fail when requests pass parameters they do
not support, regardless of whether values are bound to these parameters
or not.

\subsection{Standard Parameters}

As in SCS1, SCS2 services must support positional constraints.  They are
given as a ``cone'', i.e., a centre and a search radius.  The following
parameters -- backwards-compatible with SCS1 -- are required for that:

\begin{description}
\item[RA] -- a right ascension in degrees, understood to be ICRS at
whatever epoch the catalogue is in.
\item[DEC] -- a declination in degrees, understood to be ICRS at
whatever epoch the catalogue is in.
\item[SR] -- a search radius in degrees.
\end{description}

All these parameters are not repeatable.

When evaluating the cone constraint, services must compute spherical
distances to the cone center and only include objects closer to the
centre than SR.  For non-SCS2 clients, services should declare these
input parameters with the UCDs \ucd{pos.eq.ra}, \ucd{pos.eq.dec}, and
\ucd{pos.angDistance}, respectively.

SCS2 breaks the identity of service and catalogue that SCS1 had.
Services therefore must support the input parameter \verb|TABLE|.  This
is not repeatable, too.

The value of TABLE is a table name taken from the tableset.  Services
only offering access to a single table may accept requests without a
TABLE parameter but must accept it and raise an error if the table name
does not match what they give in their tableset.  For services
publishing multiple tables, it is an error to not pass TABLE.

All SCS2 services must support the \verb|RESPONSEFORMAT| parameter as
per DALI sect.~4.3.3.  The only required response format is VOTable,
where implementations are free to return any version with a major
version number of 1.  VOTables must be selectable as with all media
types given in DALI.  Without RESPONSEFORMAT, SCS2 services must return
VOTable.\footnote{In other words, a totally sensible and legal
implementation of RESPONSEFORMAT is to raise an error if anythin but
application/x-votable+xml or text/xml is passed to it.}

All SCS2 services must support the \verb|MAXREC| parameter as per DALI
sect.~4.3.4.

All SCS2 services must support a \verb|VERB| (for ``verbosity'')
parameter, which we keep from SCS1.  Its value must be one of 1, 2, or
3.  When the value is 1, the response should include the bare minimum of
columns that convey some basic information kept in the catalogue.  When
the value is 3, the service should return all columns making up the
underlying table.  A value of 2 requests the columns considered by the
provider to be most typically useful to the user.   It is legal to
always return the same set of columns independent of the value of VERB.

SCS2 service may support a POS parameter as defined by SIAP2
\citep{2015ivoa.spec.1223D}, except that for SCS2, tihs is a
non-repeatable parameter.

SCS2 services may support an \texttt{UPLOAD} parameter as per DALI
sect.~4.3.5.  Only a single table in VOTable format may be uploaded, and
its columsn must be named RA, DEC, and SR.  Services implementing UPLOAD
must produce a VOTable with a single result table containing the union
of a series of SCS queries with the cones defined by the rows of the
uploaded table.  Duplicate rows are not allowed.  Table uploads from
http(s) URIs and inline uploads must be supported.

\subsection{Free Parameters}

Services may offer additional query parameters to allow clients to
communicate further constraints.  To constrain float-like parameters,
use interval-typed input parameters as per DALI sect.~3.4.

Services should name the parameters like the table columns they
constrain.  If at all possible, names of free parameters should be in
all-lower case.  This is to make them reliably distinct from protocol
parameters, which are always upper case; it is still stronly discouraged
to have free parameters that after case folding clash with protocol
parameters.

\subsection{Error Conditions with Parameters}

When clients pass parameters to an SCS2 services that it does not support,
it must respond with a 400 Bad Request HTTP status code and a DALI error
message explaining which parameter caused the problem.

When clients pass multiple values for single-valued parameters, the
service must respond with a 400 Bad Request HTTP status and a DALI error
message explaining what parameter was duplicated.

When clients pass potentially conflicting parameters to an SCS2 service,
such as both RA, DEC, and SR one the one side and POS on the other, or
UPLOAD and any other positional specification, they must respond with a
400 Bad Request HTTP status and a DALI error message explaining what
conflict caused the error.


\section{Protocol Responses}

SCS2 services respond to requests as defined in DALI sect.~5.  The
following additional requirements apply:

\begin{itemize}
\item Error responses should not use a 200 HTTP status code but instead
4xx when the service sees a client error and a 5xx when the service
diagnoses an error on its side.

\item The media types text/xml and application/x-votable+xml are treated
exactly equivalent by SCS2 clients and indicate a VOTable response.

\item All responses, including errors, must be in VOTable unless the
client has requested a different RESPONSEFORMAT, in which case none of
the following constraints apply.

\item Exactly one \xmlel{FIELD}, the row identifier, must have a UCD of
\ucd{meta.id;meta.main} and must be an array of VOTable chars; this
applies even if the original catalogue uses integer identifiers.  This
must be suitable as a primary key, i.e., there must not be two rows
sharing the same row identifier.

\item Exactly one \xmlel{FIELD} must have a UCD of
\ucd{pos.eq.ra;meta.main} and must be a single floating point value
(float or double); its contents represents a right ascension in the ICRS
frame.

\item Exactly one \xmlel{FIELD} must have a UCD of
\ucd{pos.eq.dec;meta.main} and must be a single floating point value
(float or double); its contents represents a declination in the ICRS
frame.

\item All \xmlel{FIELD} elements must have a description, and for
quantities with units, the units must be given in VOUnits syntax
\citep{2014ivoa.spec.0523D}.

\end{itemize}

The epoch of the main position should be defined in a \xmlel{COOSYS} element.


\section{Registration of SCS2 Services}

SCS2 services are registered as VODataService
\citep{2021ivoa.spec.1102D} \xmlel{vs:CatalogService} records.  They must come
with a tableset; empty or missing \xmlel{colum/description} elements are
not allowed in this tableset.  SCS2 services also must declare a spatial
coverage and should declare spectral and temporal coverage as
appropriate.

Registry records of SCS2 services must have at least one capability with
the standard id
$$
\hbox{\nolinkurl{ivo://ivoa.net/SCS2#query-2.0}}
$$

Such a capability must contain exactly one interface of type
\xmlel{vs:ParamHTTP} with its \xmlel{role} set to \emph{std}.  All
accepted parameters, including the mandatory ones.  Even for them, empty
or missing param/description elements are not allowed.  All non-protocol
parameters should come with a UCD.

The capability element with the SCS-2.0 \xmlel{standardID} should be of
the type \xmlel{cs:ConeSearch} defined below.

Other capabilities are allowed; in particular, the VOSI capabilities
(which themselves are mandatory) should be declared in the registry
record as shown in DALI, sect.~2.5.  Where the SCS2-published table is
also published through a TAP services, an TAP auxiliary capability
should be given as specified in \citet{2019ivoa.spec.0520D} (DDC).

There are two major publication scenarios:

\begin{enumerate}
\item Where an SCS2 service publishes only a single table, publishers
should use a single \xmlel{vs:CatalogService} record containing the
tableset and the full capability.

\item Where a data centre has multiple tables available for SCS2, and a
generic service without extra free parameters handles them, the
preferred scenario is to have a single \xmlel{vs:CatalogService} record
declaring the capability, including the common parameters, as well as
the tableset.  In addition, for each table, there is a
\xmlel{vs:CatalogResource}-typed record with an auxiliary capability and
a relationship referencing the service record.

\end{enumerate}

Note that services supporting additional free parameters will usually
have to use pattern (1).

In the second case, the auxiliary SCS2 capability will not use the
\xmlel{cs:ConeSearch} type; at this time, we recommend a plain
\xmlel{vr:Capability}-typed capability.  Its standardID should be
$$
\hbox{\nolinkurl{ivo://ivoa.net/SCS2#query-2.0-aux}}
$$

Example registry records for both scenarios are available in
Appendix~\ref{app:examples}.


\appendix
\section{Usage Patterns (non-normative)}

In general, it is discouraged to for clients to do registry queries
constraining to SCS2 services when what their users presumably want is
data discovery; see the DDC note \citet{2019ivoa.spec.0520D} for a
discussion of service vs.~data discovery.

Hence, the typical usage pattern of a science
client\footnote{``science'' as opposed to ``infrastructure''; a
validator might have every reason to look exclusively for SCS2
capabilities, for instance.} would look like this:

\begin{itemize}
\item Perform a registry search with non-capability constraints (e.g.,
on keywords, UCDs, authors).  In the query, retrieve the available
capabilities, e.g., using \verb|ivo_string_agg| UDF.

\item If there is a capability with one of SCS2's standard ids (either
query or query-aux), offer a UI to use the service or immediately run a
simple cone search based on the the mandatory parameters (RA, DEC, SR).

\item To produce a UI, fetch the \texttt{capabilities} sibling of the
access URL discovered.  Enumerate the capability elements with an SCS2
standard identifier.

\begin{itemize}
\item If there is exactly one non-aux capability, expose
the deiclared parameters to the user\footnote{At the time of writing, we
are still missing an interoperable method to declare column statistics,
which admittedly are important to build meainingful UIs.
\citet{note:colstatnote} suggests a conceivable mechanism, and the
authors expect that a similar scheme will be adopted soon in
VODataService, at which point these statistics should be exposed to
users or APIs.}

\item If there is only one aux capability, get the access URL from that
capability's interface element, retrieve its capabilities document and
build the API or UI as described here.
\end{itemize}
\end{itemize}

Against that, clients explicitly planning to have an ``SCS2 client''
(and that is mildly discouraged) would probably want to do the
following:

\begin{itemize}
\item Perform a registry search with all normal constraint, additionally
constraining \verb|standard_id| in \verb|rr.capability| via
$$
\hbox{\verb|standard_id like 'ivo://ivoa.net/std/SCS2#query-2.%'}.$$
Also retrieve the relationships of the matches.  The wildcard is
necessary to match records having auxiliary capabilities; these contain
relevant metadata like descriptions, authors, coverage, etc.  Also
retrieve the relevant table names.

\item To avoid querying the same tables multiple times, remove any
matches with an ivoid that is also mentioned in the related resources
(these will be the main SCS services).

\item Either run simple cone queries against the services and tables
found in this way, passing to TABLE the table name(s) discovered, or let
users select individual services; in that latter case, try to build APIs
with any extra parameters as above, interpreting the service's
capabilities endpoint.
\end{itemize}

Here is an example for a RegTAP query that will yield SCS2 services
serving temperatures that would work for this kind of SCS2 UI:

\begin{lstlisting}
SELECT ivoid, access_url, res_title, res_description,
  related_ids
FROM rr.capability
  NATURAL JOIN rr.interface
  NATURAL JOIN rr.resource as b
  NATURAL LEFT OUTER JOIN (
    SELECT ivo_string_agg(related_id, '###sep###') AS related_ids
    FROM rr.relationship
    WHERE relationship_type='isservedby'
    GROUP BY ivoid) AS rels
WHERE
  standard_id LIKE 'ivo://ivoa.net/std/scs2#query-2.%'
  AND EXISTS(SELECT 1 FROM rr.table_column AS t
    WHERE t.ivoid=b.ivoid AND ucd='phys.temperature')
\end{lstlisting}

For clarity: Clients only wishing to do a plain cone search on a single
service or some hand-curated subset of SCS services do not have to
retrieve the capabilities document or do complicated elision of
collective services or do other sorts of complex service discovery.  A
functionality of the type ``query this list of SCS2 services for this
particular cone'' is of course completely in the spirit of the VO.
Still, SCS2 clients are encouraged to think more about data discovery
than about service disovery.

Also note that there is no defined way to produce a union of SCS2
results produced querying mutliple SCS2 services.  Trivially, their
table schemas will be different.  But one cannot even safely produce a
union of the guaranteed columns (main identifier, RA, and Dec); for
instance, the uniqueness requirements of the identifiers would be lost
in this way.


\section{Transition Plan (non-normative)}

\section{Example Documents}
\label{app:examples}

To help implementors, this document comes with several example documents:

\begin{itemize}
\item A sample response
\item A registry record for a one-table service with custom parameters
\item A registry record for a CatalogResource with an auxiliary SCS2
capability and the accompanying global SCS2 service declarations.
\end{itemize}

\section{Changes from Previous Versions}

No previous versions yet.
% these would be subsections "Changes from v. WD-..."
% Use itemize environments.


\bibliography{ivoatex/ivoabib,ivoatex/docrepo,local}


\end{document}
